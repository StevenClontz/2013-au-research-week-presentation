% $Header: /home/vedranm/bitbucket/beamer/solutions/generic-talks/generic-ornate-15min-45min.en.tex,v 90e850259b8b 2007/01/28 20:48:30 tantau $

\documentclass{beamer}

% This file is a solution template for:

% - Giving a talk on some subject.
% - The talk is between 15min and 45min long.
% - Style is ornate.



% Copyright 2004 by Till Tantau <tantau@users.sourceforge.net>.
%
% In principle, this file can be redistributed and/or modified under
% the terms of the GNU Public License, version 2.
%
% However, this file is supposed to be a template to be modified
% for your own needs. For this reason, if you use this file as a
% template and not specifically distribute it as part of a another
% package/program, I grant the extra permission to freely copy and
% modify this file as you see fit and even to delete this copyright
% notice. 


\mode<presentation>
{
  \usetheme{Warsaw}
  % or ...

  \setbeamercovered{transparent}
  % or whatever (possibly just delete it)
}


\usepackage[english]{babel}
% or whatever

\usepackage[latin1]{inputenc}
% or whatever

\usepackage{times}
\usepackage[T1]{fontenc}
% Or whatever. Note that the encoding and the font should match. If T1
% does not look nice, try deleting the line with the fontenc.


\usepackage{marvosym} % For \Smiley
\usepackage{verbatim} % for \verbatiminput

\title[Limited Information Strategies for Topological Games] % (optional, use only with long paper titles)
{Limited Information Strategies for Topological Games}

\subtitle
{GSC Scholars' Forum / AU Research Week 2013} % (optional)

\author%[Author, Another] % (optional, use only with lots of authors)
{Steven~Clontz}%\inst{1} \and S.~Another\inst{2}}
% - Use the \inst{?} command only if the authors have different
%   affiliation.

\institute[Auburn University] % (optional, but mostly needed)
{
  %\inst{1}%
  Department of Mathematics and Statistics\\
  Auburn University}
  %\and
  %\inst{2}%
  %Department of Theoretical Philosophy\\
  %University of Elsewhere}
% - Use the \inst command only if there are several affiliations.
% - Keep it simple, no one is interested in your street address.

\date[13-02-27] % (optional)
{February 27, 2013}

\subject{Limited Information Strategies for Topological Games}
% This is only inserted into the PDF information catalog. Can be left
% out. 



% If you have a file called "university-logo-filename.xxx", where xxx
% is a graphic format that can be processed by latex or pdflatex,
% resp., then you can add a logo as follows:

 \pgfdeclareimage[height=1cm]{university-logo}{auburn_logo.png}
 \logo{\pgfuseimage{university-logo}}



% Delete this, if you do not want the table of contents to pop up at
% the beginning of each subsection:
%\AtBeginSubsection[]
%{
%  \begin{frame}<beamer>{Outline}
%    \tableofcontents[currentsection,currentsubsection]
%  \end{frame}
%}


% If you wish to uncover everything in a step-wise fashion, uncomment
% the following command: 

%\beamerdefaultoverlayspecification{<+->}


\begin{document}

\begin{frame}
  \titlepage

  {\tiny http://www.stevenclontz.com/AURW2013/}
\end{frame}

\begin{frame}{Table of Contents}
  \tableofcontents
  % You might wish to add the option [pausesections]
\end{frame}


% Since this a solution template for a generic talk, very little can
% be said about how it should be structured. However, the talk length
% of between 15min and 45min and the theme suggest that you stick to
% the following rules:  

% - Exactly two or three sections (other than the summary).
% - At *most* three subsections per section.
% - Talk about 30s to 2min per frame. So there should be between about
%   15 and 30 frames, all told.

\section{Introduction}

\subsection{Abstract}

\begin{frame}{Abstract}%{Subtitles are optional.}
  % - A title should summarize the slide in an understandable fashion
  %   for anyone how does not follow everything on the slide itself.

  \begin{itemize}
  \item
    Many definitions of topological properties can be elegantly described in terms of a two-player ``topological game'' of countably infinite length. 
  \pause
  \item
    In a topological game, a property of the topological space being played upon is characterized by whether one player or another has a ``winning strategy'', a strategy which cannot be countered by any possible play by the opponent. 
  \pause
  \item 
    The presenter's research involves investigating several topological games from the literature for properties characterized by the existence of winning ``limited information'' strategies, such as Markov strategies which only require knowledge of the round number and only the most recent move of the opponent. 
  \end{itemize}
\end{frame}

\section{Topology \& Infinite Length Games}

\subsection{Topology}

\begin{frame}{What is Topology?}

  \begin{itemize}
  \item
    Topology is, put simply, the study of mathematical ``spaces''.
  \pause
  \item
    Most of us have learned about the (usual) topology of the real line and the $xy$-plane in calculus.
  \pause
  \item
    Topology is chiefly concerned with the ``structure'' of mathematical spaces. An example of a topological observation is that removing a point from the real line splits it into two pieces, while removing a point from the real plane does not.
  \pause
  \item
    TODO: Add pictures of $\mathbb{R}$ and $\mathbb{R}^2$, and show that removing a point splits the first but not the second.
  \end{itemize}

\end{frame}

\begin{frame}{Why study Topology?}
  \begin{itemize}
    \item 
      One of the primary uses of topology is as a toolkit for other mathematicians. Topological facts are often cited within proofs in other mathematical fields.
    \pause
    \item TODO: Consider a proof of the Fundamental Theorem of Algebra?
    \pause
    \item
      However, topology is also emerging as powerful tool in data analysis.
      \pause
      \begin{itemize}
        \item A data analysist is given a finite number of data points: ordered lists of numbers ($n$-dimensional vectors).
        \pause
        \item These points may be embedded in the Euclidean topological space $\mathbb{R}^n$: by ``fattening'' them up, we gain insight as to the structure of the source of the data.
        \pause
        \item TODO: Add picture based on paper sent by Dabbs and citation.
      \end{itemize}
  \end{itemize}
\end{frame}

\begin{frame}{What Should I Know?}
  \begin{itemize}
  \item
    For this talk, I'll stick with one familiar topological space and one (most likely) unfamiliar one.
  \pause
  \item
    When I'm talking about the \textbf{xy-plane}, I'm referring to the usual space of ordered pairs of real numbers from calculus. TODO: Add picture.
  \pause
  \item
    We'll also use another example of a topological space, known by set-theoretic topologists as the ``sequential fan''. However, let's just call it the \textbf{Milky Way space}. TODO: Add picture
  \end{itemize}
\end{frame}

\subsection{Games}

\begin{frame}{What's Game Theory?}

  \begin{itemize}
  \item
    Game theory is a powerful tool of use to anyone interested in the study of strategic decision-making: economists, biologists, logicians, political scientists...
  \pause
  \item
    Within game theory, there are two main types of two-player games: \textbf{simultaneous} and \textbf{sequential}.
  \pause
  \item
    Simultaneous games include the famous \textbf{Prisoner's Dilemma}: should a prisoner testify against his partner in exchange for a light sentence, knowing that his partner is simultaneously given the same option? TODO: Add picture
  \end{itemize}

\end{frame}

\begin{frame}{What's a Sequential Game?}
  \begin{itemize}
  \item
    However, my research is concerned with \textbf{sequential games}. Tic-tac-toe and Chess are handy examples we're all familiar with. TODO: Add pictures
  \pause
  \item
    Mathematically, we can model sequential games by tracking the decisions made by each player during each round. TODO: Add gameplay records for Tic-tac-toe and Chess.
  \end{itemize}
\end{frame}

\begin{frame}{And what's this about Infinite-Length Games?}
% cool reference http://www.logic.univie.ac.at/~ykhomski/infinitegames2010/Infinite%20Games%20Sofia.pdf
  \begin{itemize}
  \item
    A good game designer would avoid this, but mathematically we can consider games which aren't required to terminate in a victory for either player.
  \pause
  \item
    In that case, the game never ends, but as long as the players involved have a gameplan, we can consider the result of them sticking to their gameplan: the sequence of choices made by each player. TODO: Add choices for tic-tac-toe on the integer lattice.
  \pause
  \item
    If the game doesn't end, we'll have a rule to judge how each player did throughout the game, and declare a winner that way.
  \end{itemize}
\end{frame}

\begin{frame}
  \begin{itemize}
  \item 
    Example game: Player I and Player II take turns picking positive integers $2$ - $9$. A player wins as soon as if the product of all chosen numbers equals a multiple of $18$. If the game never ends, Player I wins as long as she chose $9$ at least once during the game; otherwise Player II wins.

    TODO: Have picture of two players picking numbers.
  \end{itemize}
\end{frame}

\begin{frame}
  \begin{itemize}
  \item
    While it's easy to imagine this game never ending (both players always picking $5$ would do it), we can say that Player II has a \textbf{winning strategy}:
    \pause
    \begin{itemize}
      \item Player I can't play any number besides $5$ or $7$ unless it results in a multiple of $18$ - otherwise Player II can make the multiple of $18$ on the next turn.
      \pause
      \item If Player II always plays $7$ in response to $5$ or $7$ being played by Player I, then Player I can never make a multiple of $18$ on her own.
    \end{itemize}
  \pause
  \item 
    Thus one winning strategy for Player II is to always respond with $7$ if Player I chooses $5$ or $7$, and to pick an appopriate number to make a multiple of $18$ otherwise.
  \pause
    \begin{itemize}
      \item 
        The result of any game where Player II sticks to this strategy either involves Player II making a multiple of $18$, or Player I never choosing $9$!
    \end{itemize}
  \end{itemize}
\end{frame}

\begin{frame}{So it's sort of like boxing...}
  \begin{itemize}
  \item
    You can think of these games like a \textbf{boxing match} with infinite rounds. TODO: Add Punchout screenshot
  \pause
  \item
    Each boxer takes turns swinging at each other. If the swing knocks the other guy out, that's a Win by KO.
  \pause
  \item
    If neither boxer manages to KO the other, then we turn to the judges to evaluate based on how they played throughout the entire game. One of the players must get a Win by Decision.
  \end{itemize}
\end{frame}

\section{Topological Games}

\begin{frame}{Topological Games}
  
  \begin{itemize}
    \item
      Topological games are infinite-length sequential games ``played upon'' an arbitrary topological space.
    \pause
    \item
      You can think of topological spaces as variant ``game boards'': the rules are always the same, but the available moves depend on the board we're playing on.
  \end{itemize}

\end{frame}

\subsection{Topological Darts in the $xy$-plane}

\begin{frame}{Topological Darts in the $xy$-plane}
  \begin{itemize}
    \item
      Consider a game of ``Topological Darts'' played in the $xy$-plane. During each round:
    \pause
    \begin{itemize}
      \item
        Player B places a circular dart\textbf{B}oard on the plane so that it covers the point $(0,0)$.
      \pause
      \item
        Player D responds by throwing a \textbf{D}art at the dartboard (aka picking a point on the dartboard). TODO add picture
    \end{itemize}
  \end{itemize}
\end{frame}

\begin{frame}
  \begin{itemize}
    \item
      Player B automatically wins if Player D ever misses the dartboard.
    \pause
    \item
      If the game never ends, we say Player D wins if she can show there exists a dartboard covering $(0,0)$ that she \textit{missed} during every round of the game. Otherwise Player B wins.
  \end{itemize}
\end{frame}

\begin{frame}
  Player B has a winning strategy for Topological Darts when played in the $xy$-plane. TODO: Add picture.
\end{frame}

\subsection{Topological Darts in the Milky Way Space}

\begin{frame}{Topological Darts in the Milky Way Space}
  
  \begin{itemize}
    \item
      The interior of the circular dartboards in the $xy$-plane represent topological objects known as \textbf{open sets}. What these open sets look like depend on the topological space.
    \pause
    \item
      In our so-called Milky Way Space, the dartboards / open sets placed around the point $\infty$ look like this: TODO Add picture
  \end{itemize}

\end{frame}

\begin{frame}
  
  Player D has a winning strategy for Topological Darts when played in the Milky Way Space. TODO: Add picture

\end{frame}

\subsection{So what?}

\begin{frame}{So what?}
  \begin{itemize}
    \item
      We care about these games because they provide very slick ways of describing possible structures of a topological space.
    \pause
      \begin{itemize}
        \item
          A topological space $X$ is an ``$\alpha_2$ Fr\'echet-Urysohn'' space if for each subset $A$ of $X$, and each point $x \in \overline{A}$, there exists a sequence of points in $A$ converging to $x$, and for each countable collection of sequences coverging to $x$, there is yet another sequence converging to $x$ which intersects each of these infinitely many times.
      \end{itemize}
  \end{itemize}
\end{frame}

\begin{frame}
  \begin{itemize}
    \item
      It's much simpler to say this:
    \pause
      \begin{itemize}
        \item
          A topological space $X$ is an ``$\alpha_2$ Fr\'echet-Urysohn'' space if Player D has no winning strategy in a game of Topological Darts played in $X$.
      \end{itemize}
    \pause
    \item
      So we know the $xy$-plane is ``$\alpha_2$ Fr\'echet-Urysohn'' (we found a winning strategy for Player B, so Player D doesn't have one), but the Milky Way Space isn't (we found a winning strategy for Player D).
  \end{itemize}
\end{frame}

\section{Limited Information Games}

\subsection{What are these?}

\begin{frame}{Consquences of Limited Information}
  \begin{itemize}
    \item
      My reseach is concerned with the consquences of one player having limited information in a topological game.
    \pause
    \item
      So far we've assumed both players have perfect memories. But what happens if a player can only remember the most recent move of her opponent?
  \end{itemize}
    \pause
  In the $xy$-plane, this is of no consequence to player B. TODO add picture.
\end{frame}
\begin{frame}
  But in the Milky Way Space, this lack of information hurts Player D. TODO add picture.
  \begin{itemize}
  \item
   So even though Player D has a winning perfect information strategy, Player D lacks a winning ``tactical'' strategy relying on the most recent move of the opponent.
  \end{itemize}
\end{frame}

\begin{frame}
  \begin{itemize}
    \item
      The precense or absence of perfect information strategies in a topological game characterize some structure of the space played upon - same goes for limited information strategies.
    \pause
    \begin{itemize}
      \item
        A (``countably-tight'', ``locally-compact'') topological space $X$ is \textbf{``meta-Lindel\"of''} when Player B has a winning strategy for Topological Darts played in the ``one-point compactification'' of $X$.
      \pause
      \item
        A (``countably-tight'', ``locally-compact'') topological space $X$ is \textbf{``meta-compact''} when Player B has a winning \textbf{tactical} strategy for Topological Darts played in the ``one-point compactification'' of $X$.
    \end{itemize}
  \end{itemize}
\end{frame}

\subsection{My Results}

\begin{frame}{So what have you done?}
asdf
\end{frame}

\section{Thanks / Questions?}

\begin{frame}{Thank you!}
Any questions?
\end{frame}

\end{document}


