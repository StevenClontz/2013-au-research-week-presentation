% $Header: /home/vedranm/bitbucket/beamer/solutions/generic-talks/generic-ornate-15min-45min.en.tex,v 90e850259b8b 2007/01/28 20:48:30 tantau $

\documentclass{beamer}

% This file is a solution template for:

% - Giving a talk on some subject.
% - The talk is between 15min and 45min long.
% - Style is ornate.



% Copyright 2004 by Till Tantau <tantau@users.sourceforge.net>.
%
% In principle, this file can be redistributed and/or modified under
% the terms of the GNU Public License, version 2.
%
% However, this file is supposed to be a template to be modified
% for your own needs. For this reason, if you use this file as a
% template and not specifically distribute it as part of a another
% package/program, I grant the extra permission to freely copy and
% modify this file as you see fit and even to delete this copyright
% notice. 


\mode<presentation>
{
  \usetheme{Warsaw}
  % or ...

  \setbeamercovered{transparent}
  % or whatever (possibly just delete it)
}


\usepackage[english]{babel}
% or whatever

\usepackage[latin1]{inputenc}
% or whatever

\usepackage{times}
\usepackage[T1]{fontenc}
% Or whatever. Note that the encoding and the font should match. If T1
% does not look nice, try deleting the line with the fontenc.


\usepackage{marvosym} % For \Smiley
\usepackage{verbatim} % for \verbatiminput

\title[Limited Information Strategies for Topological Games] % (optional, use only with long paper titles)
{Limited Information Strategies for Topological Games}

\subtitle
{GSC Scholars' Forum / AU Research Week 2013} % (optional)

\author%[Author, Another] % (optional, use only with lots of authors)
{Steven~Clontz}%\inst{1} \and S.~Another\inst{2}}
% - Use the \inst{?} command only if the authors have different
%   affiliation.

\institute[Auburn University] % (optional, but mostly needed)
{
  %\inst{1}%
  Department of Mathematics and Statistics\\
  Auburn University}
  %\and
  %\inst{2}%
  %Department of Theoretical Philosophy\\
  %University of Elsewhere}
% - Use the \inst command only if there are several affiliations.
% - Keep it simple, no one is interested in your street address.

\date[13-02-27] % (optional)
{February 27, 2013}

\subject{Limited Information Strategies for Topological Games}
% This is only inserted into the PDF information catalog. Can be left
% out. 



% If you have a file called "university-logo-filename.xxx", where xxx
% is a graphic format that can be processed by latex or pdflatex,
% resp., then you can add a logo as follows:

 \pgfdeclareimage[height=1cm]{university-logo}{auburn_logo.png}
 \logo{\pgfuseimage{university-logo}}



% Delete this, if you do not want the table of contents to pop up at
% the beginning of each subsection:
%\AtBeginSubsection[]
%{
%  \begin{frame}<beamer>{Outline}
%    \tableofcontents[currentsection,currentsubsection]
%  \end{frame}
%}


% If you wish to uncover everything in a step-wise fashion, uncomment
% the following command: 

%\beamerdefaultoverlayspecification{<+->}


\begin{document}

\begin{frame}
  \titlepage

  {\tiny http://www.stevenclontz.com/AURW2013/}
\end{frame}

\begin{frame}{Table of Contents}
  \tableofcontents
  % You might wish to add the option [pausesections]
\end{frame}


% Since this a solution template for a generic talk, very little can
% be said about how it should be structured. However, the talk length
% of between 15min and 45min and the theme suggest that you stick to
% the following rules:  

% - Exactly two or three sections (other than the summary).
% - At *most* three subsections per section.
% - Talk about 30s to 2min per frame. So there should be between about
%   15 and 30 frames, all told.

\section{Introduction}

\begin{frame}{Abstract}%{Subtitles are optional.}
  % - A title should summarize the slide in an understandable fashion
  %   for anyone how does not follow everything on the slide itself.

  \begin{itemize}
  \item
    Many definitions of topological properties can be elegantly described in terms of a two-player ``topological game'' of countably infinite length. 
  \pause
  \item
    In a topological game, a property of the topological space being played upon is characterized by whether one player or another has a ``winning strategy'', a strategy which cannot be countered by any possible play by the opponent. 
  \pause
  \item 
    The presenter's research involves investigating several topological games from the literature for properties characterized by the existence of winning ``limited information'' strategies, such as Markov strategies which only require knowledge of the round number and only the most recent move of the opponent. 
  \end{itemize}
\end{frame}

\section{Topology \& Infinite Length Games}

\subsection{Topology}

\begin{frame}{Topology}

  \begin{itemize}
  \item
    In this section, I'll abstractly describe topology as the study of ``closeness''.
  \pause
  \item
    I'll introduce the topology of $\mathbb{R}^2$, and the topology of a more abstract space (sequential fan, call it the Milky Way Space).
  \pause
  \item
    I'll also discuss applications, such as data analysis.
  \end{itemize}
\end{frame}

\subsection{Games}

\begin{frame}{Two Player Finite Games}

  \begin{itemize}
  \item
    In this section, I'll describe two-player simultaneous (example: Prisoner's Delimma) and sequential (example: tic-tac-toe, chess).
  \pause
  \item
    I'll model sequential games as a sequence of choices made by each player, and describe the winner as the last player to make a legal move.
  \pause
  \item
    Boxing analogy: Win by KO.
  \end{itemize}
\end{frame}

\begin{frame}{Two Player Infinite Games}

  \begin{itemize}
  \item
    Then, I can extend a finite game into an infinite-length game where there may always be legal moves to choose from for each player.
  \pause
  \item
    In that case, the game never ends, but as long as the players involved have a gameplan, we can consider the result of them sticking to said gameplan: the sequence of choices made by each player.
  \pause
  \item
    If the game didn't end, we'll have a rule to judge how each player did throughout the game, and declare a winner that way.
  \pause
  \item
    Boxing analogy: Win by Decision
  \end{itemize}
\end{frame}

\section{Example of a Topological Game}

\subsection{The Open-Point Game in $\mathbb{R}^2$}

\begin{frame}{The Open-Point Game in $\mathbb{R}^2$}
  
  \begin{itemize}
    \item
      Topological games are infinite-length games ``played upon'' an arbitrary topological space.
    \pause
    \item
      I'll walk through the rules of $Con_{O,P}(\mathbb{R}^2,\vec{0})$, using the analogy of a dartboard.
    \pause
    \item
      I'll show how $O$ has a winning strategy in this game, and introduce the notation $O \uparrow Con_{O,P}(\mathbb{R}^2,\vec{0})$
  \end{itemize}
\end{frame}

\subsection{The Open-Point Game on the Milky Way Space}

\begin{frame}{The Open-Point Game on the Milky Way Space}
  
  \begin{itemize}
    \item
      I'll compare the last game to $Con_{O,P}(MW,\infty)$.
    \pause
    \item
      I'll show the winning strategy for $P$.
  \end{itemize}

\end{frame}

\subsection{So what?}

\begin{frame}{So what?}
  \begin{itemize}
    \item
      We care about these games because they provide very slick ways of defining ``properties'' of a topological space.
    \pause
    \item
      We'll give a nasty definition of property $w$, and then claim that it's simpler to just say spaces $X$ where $O$ always has a winning strategy in the game $Con_{O,P}(X,x)$, or $O \uparrow Con_{O,P}(X,x)$.
    \pause
    \item
      So rather than going through the details of a complex definition, we can just figure out whether or not $O$ can somehow guarantee a victory in the game.
  \end{itemize}
\end{frame}

\end{document}


